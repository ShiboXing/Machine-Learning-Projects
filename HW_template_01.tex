\PassOptionsToPackage{unicode=true}{hyperref} % options for packages loaded elsewhere
\PassOptionsToPackage{hyphens}{url}
%
\documentclass[]{article}
\usepackage{lmodern}
\usepackage{amssymb,amsmath}
\usepackage{ifxetex,ifluatex}
\usepackage{fixltx2e} % provides \textsubscript
\ifnum 0\ifxetex 1\fi\ifluatex 1\fi=0 % if pdftex
  \usepackage[T1]{fontenc}
  \usepackage[utf8]{inputenc}
  \usepackage{textcomp} % provides euro and other symbols
\else % if luatex or xelatex
  \usepackage{unicode-math}
  \defaultfontfeatures{Ligatures=TeX,Scale=MatchLowercase}
\fi
% use upquote if available, for straight quotes in verbatim environments
\IfFileExists{upquote.sty}{\usepackage{upquote}}{}
% use microtype if available
\IfFileExists{microtype.sty}{%
\usepackage[]{microtype}
\UseMicrotypeSet[protrusion]{basicmath} % disable protrusion for tt fonts
}{}
\IfFileExists{parskip.sty}{%
\usepackage{parskip}
}{% else
\setlength{\parindent}{0pt}
\setlength{\parskip}{6pt plus 2pt minus 1pt}
}
\usepackage{hyperref}
\hypersetup{
            pdftitle={CS 1675 Homework: 01},
            pdfauthor={Your name here},
            pdfborder={0 0 0},
            breaklinks=true}
\urlstyle{same}  % don't use monospace font for urls
\usepackage[margin=1in]{geometry}
\usepackage{color}
\usepackage{fancyvrb}
\newcommand{\VerbBar}{|}
\newcommand{\VERB}{\Verb[commandchars=\\\{\}]}
\DefineVerbatimEnvironment{Highlighting}{Verbatim}{commandchars=\\\{\}}
% Add ',fontsize=\small' for more characters per line
\usepackage{framed}
\definecolor{shadecolor}{RGB}{248,248,248}
\newenvironment{Shaded}{\begin{snugshade}}{\end{snugshade}}
\newcommand{\AlertTok}[1]{\textcolor[rgb]{0.94,0.16,0.16}{#1}}
\newcommand{\AnnotationTok}[1]{\textcolor[rgb]{0.56,0.35,0.01}{\textbf{\textit{#1}}}}
\newcommand{\AttributeTok}[1]{\textcolor[rgb]{0.77,0.63,0.00}{#1}}
\newcommand{\BaseNTok}[1]{\textcolor[rgb]{0.00,0.00,0.81}{#1}}
\newcommand{\BuiltInTok}[1]{#1}
\newcommand{\CharTok}[1]{\textcolor[rgb]{0.31,0.60,0.02}{#1}}
\newcommand{\CommentTok}[1]{\textcolor[rgb]{0.56,0.35,0.01}{\textit{#1}}}
\newcommand{\CommentVarTok}[1]{\textcolor[rgb]{0.56,0.35,0.01}{\textbf{\textit{#1}}}}
\newcommand{\ConstantTok}[1]{\textcolor[rgb]{0.00,0.00,0.00}{#1}}
\newcommand{\ControlFlowTok}[1]{\textcolor[rgb]{0.13,0.29,0.53}{\textbf{#1}}}
\newcommand{\DataTypeTok}[1]{\textcolor[rgb]{0.13,0.29,0.53}{#1}}
\newcommand{\DecValTok}[1]{\textcolor[rgb]{0.00,0.00,0.81}{#1}}
\newcommand{\DocumentationTok}[1]{\textcolor[rgb]{0.56,0.35,0.01}{\textbf{\textit{#1}}}}
\newcommand{\ErrorTok}[1]{\textcolor[rgb]{0.64,0.00,0.00}{\textbf{#1}}}
\newcommand{\ExtensionTok}[1]{#1}
\newcommand{\FloatTok}[1]{\textcolor[rgb]{0.00,0.00,0.81}{#1}}
\newcommand{\FunctionTok}[1]{\textcolor[rgb]{0.00,0.00,0.00}{#1}}
\newcommand{\ImportTok}[1]{#1}
\newcommand{\InformationTok}[1]{\textcolor[rgb]{0.56,0.35,0.01}{\textbf{\textit{#1}}}}
\newcommand{\KeywordTok}[1]{\textcolor[rgb]{0.13,0.29,0.53}{\textbf{#1}}}
\newcommand{\NormalTok}[1]{#1}
\newcommand{\OperatorTok}[1]{\textcolor[rgb]{0.81,0.36,0.00}{\textbf{#1}}}
\newcommand{\OtherTok}[1]{\textcolor[rgb]{0.56,0.35,0.01}{#1}}
\newcommand{\PreprocessorTok}[1]{\textcolor[rgb]{0.56,0.35,0.01}{\textit{#1}}}
\newcommand{\RegionMarkerTok}[1]{#1}
\newcommand{\SpecialCharTok}[1]{\textcolor[rgb]{0.00,0.00,0.00}{#1}}
\newcommand{\SpecialStringTok}[1]{\textcolor[rgb]{0.31,0.60,0.02}{#1}}
\newcommand{\StringTok}[1]{\textcolor[rgb]{0.31,0.60,0.02}{#1}}
\newcommand{\VariableTok}[1]{\textcolor[rgb]{0.00,0.00,0.00}{#1}}
\newcommand{\VerbatimStringTok}[1]{\textcolor[rgb]{0.31,0.60,0.02}{#1}}
\newcommand{\WarningTok}[1]{\textcolor[rgb]{0.56,0.35,0.01}{\textbf{\textit{#1}}}}
\usepackage{graphicx,grffile}
\makeatletter
\def\maxwidth{\ifdim\Gin@nat@width>\linewidth\linewidth\else\Gin@nat@width\fi}
\def\maxheight{\ifdim\Gin@nat@height>\textheight\textheight\else\Gin@nat@height\fi}
\makeatother
% Scale images if necessary, so that they will not overflow the page
% margins by default, and it is still possible to overwrite the defaults
% using explicit options in \includegraphics[width, height, ...]{}
\setkeys{Gin}{width=\maxwidth,height=\maxheight,keepaspectratio}
\setlength{\emergencystretch}{3em}  % prevent overfull lines
\providecommand{\tightlist}{%
  \setlength{\itemsep}{0pt}\setlength{\parskip}{0pt}}
\setcounter{secnumdepth}{0}
% Redefines (sub)paragraphs to behave more like sections
\ifx\paragraph\undefined\else
\let\oldparagraph\paragraph
\renewcommand{\paragraph}[1]{\oldparagraph{#1}\mbox{}}
\fi
\ifx\subparagraph\undefined\else
\let\oldsubparagraph\subparagraph
\renewcommand{\subparagraph}[1]{\oldsubparagraph{#1}\mbox{}}
\fi

% set default figure placement to htbp
\makeatletter
\def\fps@figure{htbp}
\makeatother

\usepackage{etoolbox}
\makeatletter
\providecommand{\subtitle}[1]{% add subtitle to \maketitle
  \apptocmd{\@title}{\par {\large #1 \par}}{}{}
}
\makeatother

\title{CS 1675 Homework: 01}
\providecommand{\subtitle}[1]{}
\subtitle{Assigned: January 9, 2020; Due: January 17, 2020}
\author{Your name here}
\date{Submission time: January 17, 2020 at 5:00PM}

\begin{document}
\maketitle

\hypertarget{collaborators}{%
\paragraph{Collaborators}\label{collaborators}}

Include the names of your collaborators here.

\hypertarget{overview}{%
\subsection{Overview}\label{overview}}

This homework assignment focuses on the \texttt{ggplot2} and
\texttt{dplyr} packages. You will work with aesthetics and practice data
manipulation operations. Additionally, you will practice working with R
Markdown by completing the template. You can see what the rendered
document looks like at any time, by pressing the ``Knit'' button within
RStudio. You can execute a code chunk by pressing the arrow button
within that code chunk located to the upper right portion of the code
chunk within the RStudio IDE. Alternatively, you can execute a line of
code within a code chunk by selecting that line and pressing Ctrl+ENTER
(Windows) or Command+ENTER (Mac).

Completing this assignment requires filling in missing pieces of
information from existing code chunks, programming complete code chunks
from scratch, typing discussions about results, and working with LaTeX
style math formulas. A template .Rmd file is available to use as a
starting point for this homework assignment. The template is available
on CourseWeb.

\textbf{IMPORTANT:} Please pay attention to the \texttt{eval} flag
within the code chunk options. Code chunks with \texttt{eval=FALSE} will
\textbf{not} be evaluated (executed) when you Knit the document. You
\textbf{must} change the \texttt{eval} flag to be \texttt{eval=TRUE}.
This was done so that you can Knit (and thus render) the document as you
work on the assignment, without worrying about errors crashing the code
in questions you have not started. Code chunks which require you to
enter all of the required code do not set the \texttt{eval} flag. Thus,
those specific code chunks use the default option of \texttt{eval=TRUE}.

\textbf{IMPORTANT:} Each problem provides a fair amount of discussion to
help teach the \texttt{R} syntax. However, the last problem, Problem 6,
involves writing LaTeX math code. Some of the discussion was removed
from Problem 6 within this template .Rmd file because if it was included
the answers would have been given away! Therefore, please consult the
\texttt{HW\_instructions\_01.html} file for the complete discussion
associated with Problem 6.

\hypertarget{load-packages}{%
\subsection{Load packages}\label{load-packages}}

This homework assignment will use the following packages:

\begin{Shaded}
\begin{Highlighting}[]
\KeywordTok{library}\NormalTok{(dplyr)}
\KeywordTok{library}\NormalTok{(ggplot2)}
\end{Highlighting}
\end{Shaded}

\hypertarget{problem-1}{%
\subsection{Problem 1}\label{problem-1}}

The supplemental reading material
\href{https://github.com/jyurko/CS_1675_Spring_2020/blob/master/week_01/quick_r_intro.md}{A
Very Quick Intro to R} introduced \texttt{ggplot2} by creating a
histogram for the \texttt{Sepal.Length} variable within the
\texttt{iris} dataset. We will use that figure to practice modifying
color within a \texttt{ggplot2} graphic.

\hypertarget{a}{%
\subsubsection{1a)}\label{a}}

\hypertarget{problem}{%
\paragraph{PROBLEM}\label{problem}}

\textbf{Create a histogram for \texttt{Sepal.Length} from \texttt{iris}
with 6 bins, instead of the 15 bins used within the supplemental reading
material.}

\hypertarget{solution}{%
\paragraph{SOLUTION}\label{solution}}

\begin{Shaded}
\begin{Highlighting}[]
\NormalTok{iris }\OperatorTok\StringTok{ }\KeywordTok{ggplot}\NormalTok{(}\DataTypeTok{mapping =} \KeywordTok{aes}\NormalTok{(}\DataTypeTok{x =}\NormalTok{ Sepal.Length)) }\OperatorTok{+}\StringTok{ }
\StringTok{  }\KeywordTok{geom_histogram}\NormalTok{(}\DataTypeTok{bins =} \DecValTok{6}\NormalTok{)}
\end{Highlighting}
\end{Shaded}

\includegraphics{HW_template_01_files/figure-latex/solution_01a-1.pdf}

\hypertarget{b}{%
\subsubsection{1b)}\label{b}}

By default, a \texttt{ggplot2} histogram does not show the lines
associated with each bin (as in a bar graph). The histogram effectively
looks like a discretized distribution. To adjust this, we need to
override the default \texttt{fill} and \texttt{color} arguments to the
\texttt{geom\_histogram()} function. Note that within \texttt{ggplot2},
color is applied to line-like objects and points while fill is applied
to whole areas of the graph (think ``fill in an area''). Thus, you can
have substantial control over how color is used to visually present
information within a graphic.

Even though an aesthetic can be linked to a variable, some aesthetics
can be modified ``manually'' and not associated with any variables
within the dataset. We use the same type of argument, but we set that
argument outside of the \texttt{aes()} function.

\hypertarget{problem-2}{%
\paragraph{PROBLEM}\label{problem-2}}

\textbf{To see how this works, type \texttt{color\ =\ "black"} within
the \texttt{geom\_histogram()} call. Be careful about your commas!}

\hypertarget{solution-1}{%
\paragraph{SOLUTION}\label{solution-1}}

\begin{Shaded}
\begin{Highlighting}[]
\CommentTok{### your code here}
\NormalTok{iris }\OperatorTok\StringTok{ }\KeywordTok{ggplot}\NormalTok{(}\DataTypeTok{mapping =} \KeywordTok{aes}\NormalTok{(}\DataTypeTok{x =}\NormalTok{ Sepal.Length)) }\OperatorTok{+}\StringTok{ }
\StringTok{  }\KeywordTok{geom_histogram}\NormalTok{(}\DataTypeTok{bins =} \DecValTok{6}\NormalTok{, }\DataTypeTok{color =} \StringTok{'black'}\NormalTok{)}
\end{Highlighting}
\end{Shaded}

\includegraphics{HW_template_01_files/figure-latex/solution_01b-1.pdf}

\hypertarget{c}{%
\subsubsection{1c)}\label{c}}

\texttt{ggplot2} has many ``named'' colors available for use. If you
really want to fine tune your colors you are free to use the hex color
codes! In this course, we will typically stick with common colors when
we manually pick a color and/or fill.

\hypertarget{problem-3}{%
\paragraph{PROBLEM}\label{problem-3}}

\textbf{To make the difference between color and fill explicit within
the histogram, change the color to \texttt{color\ =\ "navyblue"} and
modify the histogram's fill by setting \texttt{fill\ =\ "gold"}.}

\hypertarget{solution-2}{%
\paragraph{SOLUTION}\label{solution-2}}

\begin{Shaded}
\begin{Highlighting}[]
\CommentTok{### your code here}
\NormalTok{iris }\OperatorTok\StringTok{ }\KeywordTok{ggplot}\NormalTok{(}\DataTypeTok{mapping =} \KeywordTok{aes}\NormalTok{(}\DataTypeTok{x =}\NormalTok{ Sepal.Length)) }\OperatorTok{+}\StringTok{ }
\StringTok{  }\KeywordTok{geom_histogram}\NormalTok{(}\DataTypeTok{bins =} \DecValTok{6}\NormalTok{, }\DataTypeTok{color =} \StringTok{'navyblue'}\NormalTok{, }\DataTypeTok{fill =} \StringTok{'gold'}\NormalTok{)}
\end{Highlighting}
\end{Shaded}

\includegraphics{HW_template_01_files/figure-latex/solution_01c-1.pdf}

\hypertarget{d}{%
\subsubsection{1d)}\label{d}}

We can alter the size or thickness of the lines around each bin with the
\texttt{size} argument.

\hypertarget{problem-4}{%
\paragraph{PROBLEM}\label{problem-4}}

\textbf{Set \texttt{size\ =\ 1.55} within the \texttt{geom\_histogram()}
call (using the same color scheme from Problem 1c)).}

\hypertarget{solution-3}{%
\paragraph{SOLUTION}\label{solution-3}}

\begin{Shaded}
\begin{Highlighting}[]
\NormalTok{iris }\OperatorTok\StringTok{ }\KeywordTok{ggplot}\NormalTok{(}\DataTypeTok{mapping =} \KeywordTok{aes}\NormalTok{(}\DataTypeTok{x =}\NormalTok{ Sepal.Length)) }\OperatorTok{+}\StringTok{ }
\StringTok{  }\KeywordTok{geom_histogram}\NormalTok{(}\DataTypeTok{bins =} \DecValTok{6}\NormalTok{, }\DataTypeTok{size =} \FloatTok{1.55}\NormalTok{, }\DataTypeTok{color =} \StringTok{'navyblue'}\NormalTok{, }\DataTypeTok{fill =} \StringTok{'gold'}\NormalTok{)}
\end{Highlighting}
\end{Shaded}

\includegraphics{HW_template_01_files/figure-latex/solution_01d-1.pdf}

\hypertarget{e}{%
\subsubsection{1e)}\label{e}}

Lastly, the transparency of geometric objects can be altered with the
\texttt{alpha} argument.

\hypertarget{problem-5}{%
\paragraph{PROBLEM}\label{problem-5}}

\textbf{Set the transparency to \texttt{alpha\ =\ 0.5} within the
\texttt{geom\_histogram()} call.}

\hypertarget{solution-4}{%
\paragraph{SOLUTION}\label{solution-4}}

\begin{Shaded}
\begin{Highlighting}[]
\NormalTok{iris }\OperatorTok\StringTok{ }\KeywordTok{ggplot}\NormalTok{(}\DataTypeTok{mapping =} \KeywordTok{aes}\NormalTok{(}\DataTypeTok{x =}\NormalTok{ Sepal.Length)) }\OperatorTok{+}\StringTok{ }
\StringTok{  }\KeywordTok{geom_histogram}\NormalTok{(}\DataTypeTok{bins =} \DecValTok{6}\NormalTok{, }\DataTypeTok{size =} \FloatTok{1.55}\NormalTok{, }\DataTypeTok{color =} \StringTok{'navyblue'}\NormalTok{, }\DataTypeTok{fill =} \StringTok{'gold'}\NormalTok{, }\DataTypeTok{alpha =} \FloatTok{0.5}\NormalTok{)}
\end{Highlighting}
\end{Shaded}

\includegraphics{HW_template_01_files/figure-latex/solution_01e-1.pdf}

\hypertarget{problem-2-1}{%
\subsection{Problem 2}\label{problem-2-1}}

As discussed in the supplemental reading material,
\texttt{geom\_histogram()} performs multiple operations behind the
scenes in order to generate the histogram visualization. Understanding
how those operations work will provide useful data wrangling and
manipulation practice.

\hypertarget{a-1}{%
\subsubsection{2a)}\label{a-1}}

\texttt{ggplot2} takes care of the binning and counting operations for
us when we call the histogram function. Thus, we must perform those
steps in order to recreate the histogram ``from scratch''. We will start
by learning how to discretize a continuous variable. To accomplish this,
we will need to create a vector which defines the edges of each bin
within the histogram. Within \texttt{R} nomenclature, the edges are
referred to as ``breaks''.

\hypertarget{problem-6}{%
\paragraph{PROBLEM}\label{problem-6}}

\textbf{Since we used 6 bins for our histogram, how many breaks are
associated with the histogram? Assign your answer to the variable
\texttt{num\_breaks} in the code chunk below.}

\hypertarget{solution-5}{%
\paragraph{SOLUTION}\label{solution-5}}

\begin{Shaded}
\begin{Highlighting}[]
\NormalTok{num_breaks <-}\StringTok{ }\DecValTok{7}
\end{Highlighting}
\end{Shaded}

\hypertarget{b-1}{%
\subsubsection{2b)}\label{b-1}}

Next, create the vector \texttt{hist\_breaks} by breaking up the range
spanned by \texttt{Sepal.Length} via the \texttt{seq()} function.
\texttt{seq()} has two main arguments: \texttt{from} and \texttt{to}. As
the names suggest, the \texttt{from} argument is the start of the vector
and the \texttt{to} argument is the end of the vector. In this problem,
we will set the \texttt{from} and \texttt{to} arguments such that all
observations are within the assigned discretized bounds. The third
argument to \texttt{seq()} can take multiple forms, but all serve the
same purpose: to define the length of the vector. For this problem, we
will use the \texttt{length.out} argument.

\hypertarget{problem-7}{%
\paragraph{PROBLEM}\label{problem-7}}

\textbf{Create the vector \texttt{hist\_breaks} with the \texttt{seq()}
function. Use the provided \texttt{lower\_bound} and
\texttt{upper\_bound} values in the code chunk below as the
\texttt{from} and \texttt{to} arguments, respectively. Specify the
\texttt{length.out} argument to be equal to \texttt{num\_breaks}.}

\hypertarget{solution-6}{%
\paragraph{SOLUTION}\label{solution-6}}

\begin{Shaded}
\begin{Highlighting}[]
\NormalTok{lower_bound <-}\StringTok{ }\FloatTok{4.2}

\NormalTok{upper_bound <-}\StringTok{ }\FloatTok{8.2}

\NormalTok{hist_breaks <-}\StringTok{ }\KeywordTok{seq}\NormalTok{(}\DataTypeTok{from =}\NormalTok{ lower_bound, }\DataTypeTok{to =}\NormalTok{ upper_bound, }\DataTypeTok{length.out =}\NormalTok{ num_breaks)}
\end{Highlighting}
\end{Shaded}

\hypertarget{c-1}{%
\subsubsection{2c)}\label{c-1}}

With the break positions defined, we can discretize
\texttt{Sepal.Length} into bins via the \texttt{cut()} function. To
learn about \texttt{cut()}, type \texttt{?cut} into the R console in
order to bring up the documentation. As the help page says,
\texttt{cut()} divides a numeric variable into discrete bins and
converts the result the a factor. The bin ``edges'' are based on the
values specified by the \texttt{breaks} argument to the \texttt{cut()}
function. We will practice working with \texttt{cut()} first, before
applying it to the \texttt{Sepal.Length} variable within the
\texttt{iris} dataset.

\hypertarget{problem-8}{%
\paragraph{PROBLEM}\label{problem-8}}

\textbf{In the code chunk below, assign the sequential integers
\texttt{0} through \texttt{11} to the variable \texttt{x\_dbl} using the
\texttt{:} notation. Then pass \texttt{x\_dbl} to the \texttt{cut()}
function and specify the breaks to be 0, 5, and 11 by creating a vector
containing just those three values. Assign the result of the
\texttt{cut()} function to the variable \texttt{x\_factor}. Check the
data type associated with \texttt{x\_factor}, and then print the
\texttt{x\_factor} variable to the screen.}

\hypertarget{solution-7}{%
\paragraph{SOLUTION}\label{solution-7}}

\begin{Shaded}
\begin{Highlighting}[]
\NormalTok{x_dbl <-}\StringTok{ }\DecValTok{0}\OperatorTok{:}\DecValTok{11} \CommentTok{### create vector 0 to 11}
  
\NormalTok{x_factor <-}\StringTok{ }\KeywordTok{cut}\NormalTok{(x_dbl, }\DataTypeTok{breaks =} \KeywordTok{c}\NormalTok{(}\DecValTok{0}\NormalTok{, }\DecValTok{5}\NormalTok{, }\DecValTok{11}\NormalTok{)) }\CommentTok{### set the breaks argument}

\KeywordTok{class}\NormalTok{(x_factor)}
\end{Highlighting}
\end{Shaded}

\begin{verbatim}
## [1] "factor"
\end{verbatim}

\begin{Shaded}
\begin{Highlighting}[]
\NormalTok{x_factor}
\end{Highlighting}
\end{Shaded}

\begin{verbatim}
##  [1] <NA>   (0,5]  (0,5]  (0,5]  (0,5]  (0,5]  (5,11] (5,11] (5,11] (5,11]
## [11] (5,11] (5,11]
## Levels: (0,5] (5,11]
\end{verbatim}

\hypertarget{d-1}{%
\subsubsection{2d)}\label{d-1}}

The unique values associated with a factor variable are displayed in the
line below the individual elements of the vector. In \texttt{R}
nomenclature, factor unique values are referred to as ``levels''. In
your answer to Problem 2c) however, you should notice that the first
printed element reads \texttt{\textless{}NA\textgreater{}}. That's
because the discretized intervals are setup to be (, {]} (which reads as
``from but excluding lower value, to and including upper value''). We
can modify this structure to include the lowest (minimum) value in the
dataset by setting \texttt{include.lowest\ =\ TRUE} in the call to the
\texttt{cut()} function.

\hypertarget{problem-9}{%
\paragraph{PROBLEM}\label{problem-9}}

\textbf{Reperform the \texttt{cut()} call on \texttt{x\_dbl}, but this
time set \texttt{include.lowest\ =\ TRUE}. Assign the result to the
vector \texttt{x\_factor\_2} and then print the result to the screen.}

\hypertarget{solution-8}{%
\paragraph{SOLUTION}\label{solution-8}}

\begin{Shaded}
\begin{Highlighting}[]
\NormalTok{x_factor_}\DecValTok{2}\NormalTok{ <-}\StringTok{ }\KeywordTok{cut}\NormalTok{(x_dbl, }\DataTypeTok{breaks =} \KeywordTok{c}\NormalTok{(}\DecValTok{0}\NormalTok{, }\DecValTok{5}\NormalTok{, }\DecValTok{11}\NormalTok{), }\DataTypeTok{include.lowest =} \OtherTok{TRUE}\NormalTok{) }\CommentTok{### set the breaks and add the new argument}

\NormalTok{x_factor_}\DecValTok{2}
\end{Highlighting}
\end{Shaded}

\begin{verbatim}
##  [1] [0,5]  [0,5]  [0,5]  [0,5]  [0,5]  [0,5]  (5,11] (5,11] (5,11] (5,11]
## [11] (5,11] (5,11]
## Levels: [0,5] (5,11]
\end{verbatim}

\hypertarget{e-1}{%
\subsubsection{2e)}\label{e-1}}

We will now modify the \texttt{iris} dataset by discretizing
\texttt{Sepal.Length} via the \texttt{cut()} function. We will use the
\texttt{dplyr::mutate()} ``action verb'' to execute this task. As the
function name suggests, \texttt{mutate()} modifies the dataset by
creating new variables. The generic syntax is
\texttt{mutate(\textless{}new\ variable\textgreater{}\ =\ \textless{}set\ of\ actions\textgreater{})}.
We can perform simple actions to create a new variable, or execute a
complex set of tasks. \texttt{dplyr::mutate()} is a general ``wrapper''
for performing those operations.

\hypertarget{problem-10}{%
\paragraph{PROBLEM}\label{problem-10}}

\textbf{Create the new discretized variable,
\texttt{sepal\_length\_bin}, by setting the \texttt{breaks} argument
within the cut function to be \texttt{hist\_breaks} and set the
\texttt{include.lowest} argument to be \texttt{TRUE}. Assign the
modified dataset to the variable \texttt{my\_iris}.}

\hypertarget{solution-9}{%
\paragraph{SOLUTION}\label{solution-9}}

\begin{Shaded}
\begin{Highlighting}[]
\NormalTok{my_iris <-}\StringTok{ }\NormalTok{iris }\OperatorTok\StringTok{ }
\StringTok{  }\KeywordTok{mutate}\NormalTok{(}\DataTypeTok{sepal_length_bin =} \KeywordTok{cut}\NormalTok{(iris}\OperatorTok{$}\NormalTok{Sepal.Length, }\DataTypeTok{breaks =}\NormalTok{ hist_breaks, }\DataTypeTok{include.lowest =} \OtherTok{TRUE}\NormalTok{))}
\end{Highlighting}
\end{Shaded}

\hypertarget{problem-3-1}{%
\subsection{Problem 3}\label{problem-3-1}}

Our new dataset, \texttt{my\_iris}, contains an additional variable
compared to the ``base'' \texttt{iris}. We will now practice
manipulating the dataset based on grouping with this new variable.

\hypertarget{a-2}{%
\subsubsection{3a)}\label{a-2}}

Let's get an overview of the newly created variable,
\texttt{sepal\_length\_bin}.

\hypertarget{problem-11}{%
\paragraph{PROBLEM}\label{problem-11}}

\textbf{Use the \texttt{select()} action verb to isolate the
\texttt{sepal\_length\_bin} variable from the other variables and pipe
the result to the \texttt{summary()} function.}

\hypertarget{solution-10}{%
\paragraph{SOLUTION}\label{solution-10}}

\begin{Shaded}
\begin{Highlighting}[]
\KeywordTok{select}\NormalTok{(my_iris, }\StringTok{'sepal_length_bin'}\NormalTok{) }\OperatorTok\StringTok{ }\KeywordTok{summary}\NormalTok{()}
\end{Highlighting}
\end{Shaded}

\begin{verbatim}
##     sepal_length_bin
##  [4.2,4.87] :16     
##  (4.87,5.53]:43     
##  (5.53,6.2] :36     
##  (6.2,6.87] :38     
##  (6.87,7.53]:11     
##  (7.53,8.2] : 6
\end{verbatim}

\hypertarget{b-2}{%
\subsubsection{3b)}\label{b-2}}

Because \texttt{sepal\_length\_bin} is a factor variable, we can easily
get the unique values (levels) with the \texttt{levels()} function.

\hypertarget{problem-12}{%
\paragraph{PROBLEM}\label{problem-12}}

\textbf{Pass the \texttt{sepal\_length\_bin} vector into the
\texttt{levels()} function, by accessing it with the \texttt{\$}
operator from within \texttt{my\_iris}.}

\hypertarget{solution-11}{%
\paragraph{SOLUTION}\label{solution-11}}

\begin{Shaded}
\begin{Highlighting}[]
\KeywordTok{levels}\NormalTok{(my_iris}\OperatorTok{$}\NormalTok{sepal_length_bin)}
\end{Highlighting}
\end{Shaded}

\begin{verbatim}
## [1] "[4.2,4.87]"  "(4.87,5.53]" "(5.53,6.2]"  "(6.2,6.87]"  "(6.87,7.53]"
## [6] "(7.53,8.2]"
\end{verbatim}

\hypertarget{c-2}{%
\subsubsection{3c)}\label{c-2}}

In general, we can use the \texttt{unique()} function to identify the
unique values contained within a vector irregardless of whether that
vector is a \texttt{"numeric"}, \texttt{"character"}, or a factor
variable. There are several important differences between factor levels
and unique values, but we will not be too concerned with those
differences at the moment.

\hypertarget{problem-13}{%
\paragraph{PROBLEM}\label{problem-13}}

\textbf{Call \texttt{unique()} on \texttt{sepal\_length\_bin} by
accessing the variable with the \texttt{\$} operator again.}

\hypertarget{solution-12}{%
\paragraph{SOLUTION}\label{solution-12}}

\begin{Shaded}
\begin{Highlighting}[]
\KeywordTok{unique}\NormalTok{(my_iris}\OperatorTok{$}\NormalTok{sepal_length_bin)}
\end{Highlighting}
\end{Shaded}

\begin{verbatim}
## [1] (4.87,5.53] [4.2,4.87]  (5.53,6.2]  (6.87,7.53] (6.2,6.87]  (7.53,8.2] 
## 6 Levels: [4.2,4.87] (4.87,5.53] (5.53,6.2] (6.2,6.87] ... (7.53,8.2]
\end{verbatim}

\hypertarget{d-2}{%
\subsubsection{3d)}\label{d-2}}

As you should have seen in the previous set of results, the
\texttt{levels()} and \texttt{unique()} functions return vectors.
However, if we use the \texttt{dplyr} function \texttt{distinct()}, we
will return a \texttt{data.frame} containing the unique values
associated with the variable(s). We tell \texttt{distinct} which
variable(s) to focus on using non-standard evaluation.

\hypertarget{problem-14}{%
\paragraph{PROBLEM}\label{problem-14}}

\textbf{In the code chunk below, pipe \texttt{my\_iris} to the
\texttt{distinct()} function and specify the variable to be
\texttt{sepal\_length\_bin}.}

\hypertarget{solution-13}{%
\paragraph{SOLUTION}\label{solution-13}}

\begin{Shaded}
\begin{Highlighting}[]
\NormalTok{my_iris }\OperatorTok\StringTok{ }
\StringTok{  }\KeywordTok{distinct}\NormalTok{(sepal_length_bin)}
\end{Highlighting}
\end{Shaded}

\hypertarget{e-2}{%
\subsubsection{3e)}\label{e-2}}

\texttt{distinct()} is a useful function, but it does not provide any
other information about the variables. The \texttt{count()} function,
however, returns the number of rows asssociated with each unique value.
The number of rows (or the count) are stored in an automatically created
additional column named \texttt{n}.

\hypertarget{problem-15}{%
\paragraph{PROBLEM}\label{problem-15}}

\textbf{In the code chunk below, apply the \texttt{count()} function
instead of the \texttt{distinct()} function to the
\texttt{sepal\_length\_bin} variable.}

\hypertarget{solution-14}{%
\paragraph{SOLUTION}\label{solution-14}}

\begin{Shaded}
\begin{Highlighting}[]
\CommentTok{### your code here}
\NormalTok{my_iris }\OperatorTok\StringTok{ }\KeywordTok{count}\NormalTok{(sepal_length_bin)}
\end{Highlighting}
\end{Shaded}

\begin{verbatim}
## # A tibble: 6 x 2
##   sepal_length_bin     n
##   <fct>            <int>
## 1 [4.2,4.87]          16
## 2 (4.87,5.53]         43
## 3 (5.53,6.2]          36
## 4 (6.2,6.87]          38
## 5 (6.87,7.53]         11
## 6 (7.53,8.2]           6
\end{verbatim}

\hypertarget{f}{%
\subsubsection{3f)}\label{f}}

We have sufficient information with the \texttt{count()} function result
to make a bar graph in the style of the histogram we are interested in.

\hypertarget{problem-16}{%
\paragraph{PROBLEM}\label{problem-16}}

\textbf{Pipe the result of the \texttt{count()} function into the
\texttt{ggplot()} function. Set the \texttt{x} and \texttt{y} aesthetics
to be \texttt{sepal\_length\_bin} and \texttt{n}, respectively. Then
call \texttt{geom\_bar()}. However, by default \texttt{geom\_bar()} will
not work with this setup. As with \texttt{geom\_histogram()},
\texttt{geom\_bar()} performs a grouping and counting operations behind
the scenes. To override this default behavior, we must modify the
\texttt{stat} argument to be \texttt{stat\ =\ "identity"}. The code
chunk below already has the correct \texttt{geom\_bar()} call. Fill in
the missing code to create the visualization of interest.}

\hypertarget{solution-15}{%
\paragraph{SOLUTION}\label{solution-15}}

\begin{Shaded}
\begin{Highlighting}[]
\CommentTok{### complete the ggplot call below}
\NormalTok{my_iris }\OperatorTok\StringTok{ }
\StringTok{  }\KeywordTok{count}\NormalTok{(sepal_length_bin) }\OperatorTok\StringTok{ }
\StringTok{  }\KeywordTok{ggplot}\NormalTok{(}\DataTypeTok{mapping =} \KeywordTok{aes}\NormalTok{(}\DataTypeTok{x =}\NormalTok{ sepal_length_bin, }\DataTypeTok{y =}\NormalTok{ n)) }\OperatorTok{+}
\StringTok{  }\KeywordTok{geom_bar}\NormalTok{(}\DataTypeTok{stat =} \StringTok{"identity"}\NormalTok{)}
\end{Highlighting}
\end{Shaded}

\includegraphics{HW_template_01_files/figure-latex/solution_03f-1.pdf}

\begin{Shaded}
\begin{Highlighting}[]
  \CommentTok{#n() function returns the number observations of the current group}
\end{Highlighting}
\end{Shaded}

\hypertarget{g}{%
\subsubsection{3g)}\label{g}}

The resulting bar graph in Problem 3f) looks different from the
histogram visualized in Problem 1).

\hypertarget{problem-17}{%
\paragraph{PROBLEM}\label{problem-17}}

\textbf{Why is the x-axis setup differently from the histogram in
Problem 1)?}

\hypertarget{solution-16}{%
\paragraph{SOLUTION}\label{solution-16}}

Here we map the sepal\_length\_bin to the x-axis. `sepal\_length\_bin'
was created by cut() function with 3 breakpoints, generating four
possible values. Since we didn't set a arbitrary bin count, naturally we
will have 4 intervals, as in 4 bins in our 3f) histogram.

In problem 1, we map the Sepal.length to the x-axis, which has more
unique values than sepal\_length\_bin. Also, we set the bin count to an
arbitrary number. Thus, the x-axes of the histograms look differently in
these two problems.

\hypertarget{problem-4-1}{%
\subsection{Problem 4}\label{problem-4-1}}

The \texttt{count()} function is useful for quickly grouping and
counting by one or multiple variables. However, that is all the function
is intended to do. In order to perform other calculations we need to use
the \texttt{group\_by()} function in conjunction with a function like
\texttt{summarize()}. In this problem, you will learn about these very
useful and important operations.

\hypertarget{a-3}{%
\subsubsection{4a)}\label{a-3}}

The grouping variables are specified using non-standard evaluation in
the \texttt{group\_by()} call. Thus, to group by a variable use the
following syntax:
\texttt{group\_by(\textless{}variable\ name\textgreater{})}. Grouping by
3 variables is just a straight forward extension of this:
\texttt{group\_by(\textless{}variable\ 1\textgreater{},\ \textless{}variable\ 2\textgreater{},\ \textless{}variable\ 3\textgreater{})}.

\hypertarget{problem-18}{%
\paragraph{PROBLEM}\label{problem-18}}

\textbf{Pipe \texttt{my\_iris} into the \texttt{group\_by()} function
and specify the grouping variable to be \texttt{sepal\_length\_bin}.}

\hypertarget{solution-17}{%
\paragraph{SOLUTION}\label{solution-17}}

\begin{Shaded}
\begin{Highlighting}[]
\NormalTok{my_iris }\OperatorTok\StringTok{ }
\StringTok{  }\KeywordTok{group_by}\NormalTok{(sepal_length_bin) }\CommentTok{#grouping doesn't change how data looks}
\end{Highlighting}
\end{Shaded}

\begin{verbatim}
## # A tibble: 150 x 6
## # Groups:   sepal_length_bin [6]
##    Sepal.Length Sepal.Width Petal.Length Petal.Width Species sepal_length_bin
##           <dbl>       <dbl>        <dbl>       <dbl> <fct>   <fct>           
##  1          5.1         3.5          1.4         0.2 setosa  (4.87,5.53]     
##  2          4.9         3            1.4         0.2 setosa  (4.87,5.53]     
##  3          4.7         3.2          1.3         0.2 setosa  [4.2,4.87]      
##  4          4.6         3.1          1.5         0.2 setosa  [4.2,4.87]      
##  5          5           3.6          1.4         0.2 setosa  (4.87,5.53]     
##  6          5.4         3.9          1.7         0.4 setosa  (4.87,5.53]     
##  7          4.6         3.4          1.4         0.3 setosa  [4.2,4.87]      
##  8          5           3.4          1.5         0.2 setosa  (4.87,5.53]     
##  9          4.4         2.9          1.4         0.2 setosa  [4.2,4.87]      
## 10          4.9         3.1          1.5         0.1 setosa  (4.87,5.53]     
## # ... with 140 more rows
\end{verbatim}

\hypertarget{b-3}{%
\subsubsection{4b)}\label{b-3}}

The result after applying the grouping operation appears to be just the
original dataset. However, this new dataset ``knows'' the grouping
structure. Thus, any additional operations we apply will be applied to
the grouped dataset. When we wish to summarize based on the grouping
structure, we pipe the \texttt{group\_by()} result into the
\texttt{summarize()} function. Calculations within the
\texttt{summarize()} function are performed with syntax like we used
with the \texttt{mutate()} function.

The code chunk below shows how the number of observations per
\texttt{sepal\_length\_bin} value are computed using the \texttt{n()}
function within the \texttt{summarize()} call.

\hypertarget{problem-19}{%
\paragraph{PROBLEM}\label{problem-19}}

\textbf{Complete the code chunk below by naming this summary variable
\texttt{num\_obs}. How do the resulting counts compare with the result
from \texttt{count()}?}

\emph{Note}: the \texttt{n()} function is an example of a function
without any input arguments. The syntax for calling such a function
still requires the \texttt{()} after the function name. This syntax
instructs the \texttt{R} interpreter that you are using a function named
\texttt{n} and not a variable named \texttt{n}.

\hypertarget{solution-18}{%
\paragraph{SOLUTION}\label{solution-18}}

\begin{Shaded}
\begin{Highlighting}[]
\NormalTok{my_iris }\OperatorTok\StringTok{ }
\StringTok{  }\KeywordTok{group_by}\NormalTok{(sepal_length_bin) }\OperatorTok\StringTok{ }
\StringTok{  }\KeywordTok{summarise}\NormalTok{( }\DataTypeTok{num_obs =} \KeywordTok{n}\NormalTok{())}
\end{Highlighting}
\end{Shaded}

\begin{verbatim}
## # A tibble: 6 x 2
##   sepal_length_bin num_obs
##   <fct>              <int>
## 1 [4.2,4.87]            16
## 2 (4.87,5.53]           43
## 3 (5.53,6.2]            36
## 4 (6.2,6.87]            38
## 5 (6.87,7.53]           11
## 6 (7.53,8.2]             6
\end{verbatim}

The n() function gets the same result as the count() function. The
difference between the two is simply how we format them. n() can only be
used in certain dplyr functions like summarise(), which returns the
number of observation in each group. If we want to achieve the same
result with count(), we simply pipe the result of the group\_by() to the
count() function.

\hypertarget{c-3}{%
\subsubsection{4c)}\label{c-3}}

We can do more than just counting in the \texttt{summarize()} call. We
can calculate averages, standard deviations, quantiles, and even call
complex modeling functions. Think of \texttt{summarize()} as a versatile
wrapper function for controlling operations applied to a grouped
dataset.

\hypertarget{problem-20}{%
\paragraph{PROBLEM}\label{problem-20}}

\textbf{Complete the code chunk below to calculate the average, minimum,
and maximum \texttt{Sepal.Length} values for each
\texttt{sepal\_length\_bin} level. In \texttt{R}, the average is
calculated with the \texttt{mean()} function. The minimum and maximum
values are calculated with the \texttt{min()} and \texttt{max()}
functions, respectively. The grouped and summarized dataset has been
named \texttt{iris\_group} and is printed to the screen for review.}

\hypertarget{solution-19}{%
\paragraph{SOLUTION}\label{solution-19}}

\begin{Shaded}
\begin{Highlighting}[]
\NormalTok{iris_group <-}\StringTok{ }\NormalTok{my_iris }\OperatorTok\StringTok{ }
\StringTok{  }\KeywordTok{group_by}\NormalTok{(sepal_length_bin) }\OperatorTok\StringTok{ }
\StringTok{  }\KeywordTok{summarise}\NormalTok{(}\DataTypeTok{num_obs =} \KeywordTok{n}\NormalTok{(),}
            \DataTypeTok{avg_sepal_length =} \KeywordTok{mean}\NormalTok{(Sepal.Length),}
            \DataTypeTok{min_sepal_length =} \KeywordTok{min}\NormalTok{(Sepal.Length),}
            \DataTypeTok{max_sepal_length =} \KeywordTok{max}\NormalTok{(Sepal.Length))}

\NormalTok{iris_group}
\end{Highlighting}
\end{Shaded}

\begin{verbatim}
## # A tibble: 6 x 5
##   sepal_length_bin num_obs avg_sepal_length min_sepal_length max_sepal_length
##   <fct>              <int>            <dbl>            <dbl>            <dbl>
## 1 [4.2,4.87]            16             4.61              4.3              4.8
## 2 (4.87,5.53]           43             5.17              4.9              5.5
## 3 (5.53,6.2]            36             5.84              5.6              6.1
## 4 (6.2,6.87]            38             6.47              6.2              6.8
## 5 (6.87,7.53]           11             7.09              6.9              7.4
## 6 (7.53,8.2]             6             7.72              7.6              7.9
\end{verbatim}

\hypertarget{d-3}{%
\subsubsection{4d)}\label{d-3}}

We can use the grouped and summarized dataset to create a simple
histogram which introduces the \texttt{geom\_point()} and
\texttt{geom\_linerange()} geoms. \texttt{geom\_point()} creates a basic
scatter plot between two variables mapped to the \texttt{x} and
\texttt{y} aesthetics.

\hypertarget{problem-21}{%
\paragraph{PROBLEM}\label{problem-21}}

\textbf{In the code chunk below, assign \texttt{avg\_sepal\_length} to
the \texttt{x} aesthetic within the parent \texttt{ggplot()} call. Then,
set \texttt{num\_obs} to the \texttt{y} aesthetic within the
\texttt{geom\_point()} call. Lastly, assign the \texttt{size} argument
within \texttt{geom\_point()} to be 4.}

\hypertarget{solution-20}{%
\paragraph{SOLUTION}\label{solution-20}}

\begin{Shaded}
\begin{Highlighting}[]
\NormalTok{iris_group }\OperatorTok\StringTok{ }
\StringTok{  }\KeywordTok{ggplot}\NormalTok{(}\DataTypeTok{mapping =} \KeywordTok{aes}\NormalTok{(}\DataTypeTok{x =}\NormalTok{ avg_sepal_length)) }\OperatorTok{+}
\StringTok{  }\KeywordTok{geom_point}\NormalTok{(}\DataTypeTok{mapping =} \KeywordTok{aes}\NormalTok{(}\DataTypeTok{y =}\NormalTok{ num_obs),}
             \DataTypeTok{size =} \DecValTok{4}\NormalTok{)}
\end{Highlighting}
\end{Shaded}

\includegraphics{HW_template_01_files/figure-latex/solution_04d-1.pdf}

\hypertarget{e-3}{%
\subsubsection{4e)}\label{e-3}}

A scatter plot can be very useful graphic, but it can be difficult to
visualize the distribution shape we are trying to represent in this
example. To aid in that visualization, we will add vertical lines to our
graphic via the \texttt{geom\_linerange()} function. In
\texttt{geom\_linerange()} we must specify an \texttt{x} aesthetic and
two separate y-axis aesthetics, \texttt{ymin} and \texttt{ymax}. As
their names suggest, \texttt{ymin} displays the minimum value on the
y-axis and \texttt{ymax} shows the maximum value on the y-axis.
\texttt{geom\_linerange()} then draws a vertical line connecting the
y-axis aesthetics. Since we want to use the vertical lines to represent
a histogram, set \texttt{ymin} to be 0 and \texttt{ymax} to equal
\texttt{num\_obs}. Thus, a vertical line will provide a comparable
visualization as the bar chart from Problem 3.

\hypertarget{problem-22}{%
\paragraph{PROBLEM}\label{problem-22}}

\textbf{Complete the code chunk below by correctly assigning the y-axis
aesthetics.}

\emph{Note}: this style of figure is sometimes referred to as a stem
plot. They are useful in certain contexts, but when creating histograms
I prefer the default ggplot2 \texttt{geom\_histogram()} or
\texttt{geom\_freqpoly()} style. We used the stem plot in this problem
to introduce additional geoms and to continue demonstrating the
\texttt{group\_by()} and \texttt{summarize()} functions.

\hypertarget{solution-21}{%
\paragraph{SOLUTION}\label{solution-21}}

\begin{Shaded}
\begin{Highlighting}[]
\NormalTok{iris_group }\OperatorTok\StringTok{ }
\StringTok{  }\KeywordTok{ggplot}\NormalTok{(}\DataTypeTok{mapping =} \KeywordTok{aes}\NormalTok{(}\DataTypeTok{x =}\NormalTok{ avg_sepal_length)) }\OperatorTok{+}
\StringTok{  }\KeywordTok{geom_linerange}\NormalTok{(}\DataTypeTok{mapping =} \KeywordTok{aes}\NormalTok{(}\DataTypeTok{ymin =} \DecValTok{0}\NormalTok{,}
                               \DataTypeTok{ymax =}\NormalTok{ num_obs)) }\OperatorTok{+}
\StringTok{  }\KeywordTok{geom_point}\NormalTok{(}\DataTypeTok{mapping =} \KeywordTok{aes}\NormalTok{(}\DataTypeTok{y =}\NormalTok{ num_obs),}
             \DataTypeTok{size =} \DecValTok{4}\NormalTok{)}
\end{Highlighting}
\end{Shaded}

\includegraphics{HW_template_01_files/figure-latex/solution_04e-1.pdf}

\hypertarget{f-1}{%
\subsubsection{4f)}\label{f-1}}

Let's now compare our manual histogram to ggplot's histogram. Our stem
plot simplified histogram visualized the counts at the average
\texttt{Sepal.Length} value within each bin. \texttt{geom\_freqpoly()}
plots the counts at the bin midpoint. Thus, assuming the bin center and
the (empirical) average within a bin are similar, we should have an
``apples-to-apples'' comparison between \texttt{geom\_freqpoly()} and
our stem plot.

The graphic we wish to make does not just contain multiple layers but
multiple datasets! The reason why is because we have to allow
\texttt{geom\_freqpoly()} to operate on the original non-grouped and
non-summarized dataset. We can override the data associated with a geom
by supplying an alternative dataset to the \texttt{data} argument. In
this way, we can create complex figures which make use of various
datasets.

\hypertarget{problem-23}{%
\paragraph{PROBLEM}\label{problem-23}}

\textbf{Complete the code chunk below by setting the \texttt{data}
argument to \texttt{geom\_freqpoly()} to be \texttt{my\_iris}. Specify
the appropriate variable to the \texttt{x} aesthetic within the
\texttt{geom\_freqpoly()} call. Assess how well our manual histogram
performed relative to the native ggplot histogram. Were we close? What
are the differences?}

\hypertarget{solution-22}{%
\paragraph{SOLUTION}\label{solution-22}}

\begin{Shaded}
\begin{Highlighting}[]
\NormalTok{iris_group }\OperatorTok\StringTok{ }
\StringTok{  }\KeywordTok{ggplot}\NormalTok{(}\DataTypeTok{mapping =} \KeywordTok{aes}\NormalTok{(}\DataTypeTok{x =}\NormalTok{ avg_sepal_length)) }\OperatorTok{+}
\StringTok{  }\KeywordTok{geom_freqpoly}\NormalTok{(}\DataTypeTok{data =}\NormalTok{ my_iris,}
                \DataTypeTok{mapping =} \KeywordTok{aes}\NormalTok{(}\DataTypeTok{x =}\NormalTok{Sepal.Length),}
                \DataTypeTok{bins =} \DecValTok{6}\NormalTok{, }\DataTypeTok{color =} \StringTok{"red"}\NormalTok{, }\DataTypeTok{size =} \FloatTok{1.15}\NormalTok{) }\OperatorTok{+}
\StringTok{  }\KeywordTok{geom_linerange}\NormalTok{(}\DataTypeTok{mapping =} \KeywordTok{aes}\NormalTok{(}\DataTypeTok{ymin =} \DecValTok{0}\NormalTok{,}
                               \DataTypeTok{ymax =}\NormalTok{ num_obs)) }\OperatorTok{+}
\StringTok{  }\KeywordTok{geom_point}\NormalTok{(}\DataTypeTok{mapping =} \KeywordTok{aes}\NormalTok{(}\DataTypeTok{y =}\NormalTok{ num_obs),}
             \DataTypeTok{size =} \DecValTok{4}\NormalTok{)}
\end{Highlighting}
\end{Shaded}

\includegraphics{HW_template_01_files/figure-latex/solution_04f_a-1.pdf}

Our histogram shows the increases and decreases of the number of
observations. The change in the frequency of the data might not be clear
in the native historgram. With the line segments in the histogram, we
can more easily envision the distrbution of the data set. On the other
hand, in the native histogram, many outliers and the uneven bars hinders
our ability to make sense of the distribution.

Additionally, the average lengths that have the highest frequencies are
shown, in the native histogram we can hardly tell what where mode of the
distribution is.

\hypertarget{problem-5-1}{%
\subsection{Problem 5}\label{problem-5-1}}

So far, we have used several different types of geometric objects. In
this problem, we will introduce another very important geom, the
boxplot, which provides a quick visual display of useful summary
statistics for continuous variables (\texttt{"numeric"}s). Compared with
the histogram which focuses on displaying the \emph{shape} of the
distribution, the boxplot allows us to visually relate the median with
the 25th and 75th quantiles, as well as outliers. We get an idea about
the central tendency of the variable, as well as a rough guide on the
``meaningful'' range.

To demonstrate the usefulness of the boxplot, we will use the
\texttt{diamonds} dataset from \texttt{ggplot2}.

\hypertarget{a-4}{%
\subsubsection{5a)}\label{a-4}}

\hypertarget{problem-24}{%
\paragraph{PROBLEM}\label{problem-24}}

\textbf{Pipe \texttt{diamonds} into the \texttt{glimpse()} function to
display the dimensions and datatypes associated with the variables
within the dataset.}

\hypertarget{solution-23}{%
\paragraph{SOLUTION}\label{solution-23}}

\begin{Shaded}
\begin{Highlighting}[]
\NormalTok{diamonds }\OperatorTok\StringTok{ }\KeywordTok{glimpse}\NormalTok{()}
\end{Highlighting}
\end{Shaded}

\begin{verbatim}
## Observations: 53,940
## Variables: 10
## $ carat   <dbl> 0.23, 0.21, 0.23, 0.29, 0.31, 0.24, 0.24, 0.26, 0.22, 0.23,...
## $ cut     <ord> Ideal, Premium, Good, Premium, Good, Very Good, Very Good, ...
## $ color   <ord> E, E, E, I, J, J, I, H, E, H, J, J, F, J, E, E, I, J, J, J,...
## $ clarity <ord> SI2, SI1, VS1, VS2, SI2, VVS2, VVS1, SI1, VS2, VS1, SI1, VS...
## $ depth   <dbl> 61.5, 59.8, 56.9, 62.4, 63.3, 62.8, 62.3, 61.9, 65.1, 59.4,...
## $ table   <dbl> 55, 61, 65, 58, 58, 57, 57, 55, 61, 61, 55, 56, 61, 54, 62,...
## $ price   <int> 326, 326, 327, 334, 335, 336, 336, 337, 337, 338, 339, 340,...
## $ x       <dbl> 3.95, 3.89, 4.05, 4.20, 4.34, 3.94, 3.95, 4.07, 3.87, 4.00,...
## $ y       <dbl> 3.98, 3.84, 4.07, 4.23, 4.35, 3.96, 3.98, 4.11, 3.78, 4.05,...
## $ z       <dbl> 2.43, 2.31, 2.31, 2.63, 2.75, 2.48, 2.47, 2.53, 2.49, 2.39,...
\end{verbatim}

\hypertarget{b-4}{%
\subsubsection{5b)}\label{b-4}}

The variable \texttt{price} gives the cost of a diamond in US dollars,
while the other variables provide attributes associated with a diamond.
A natural question to ask then is: \emph{What variables influence the
price?} If you have purchased a diamond before, you may have heard about
the ``4 C's'': cut, color, clarity, and carat. We will explore the
behavior of \texttt{price} with respect to these 4 variables.

The first three of these are all factors (the \texttt{"ord"} data type
is a special ordered factor which as the name states has a natural
ordering of the categorical levels) within the \texttt{diamonds}
dataset, while \texttt{carat} is a \texttt{"numeric"} variable. With a
boxplot, we group a continuous variable based on a categorical variable,
and display summary statistics associated with the continuous variable
for each categorical level. Therefore, the boxplot geom is another
geometric object which performs multiple operations behind the scenes.
If you have not worked boxplots before, I recommend Chapter 7 of the
\href{https://r4ds.had.co.nz/}{R for Data Science} book.

Let's start by visualizing the relationship between \texttt{price} and
\texttt{color}.

\hypertarget{problem-25}{%
\paragraph{PROBLEM}\label{problem-25}}

\textbf{Pipe \texttt{diamonds} into \texttt{ggplot()} and set the
\texttt{x} and \texttt{y} aesthetics to \texttt{color} and
\texttt{price}, respectively. Then, call \texttt{geom\_boxplot()}. What
conclusion would you draw based on the resulting figure?}

\hypertarget{solution-24}{%
\paragraph{SOLUTION}\label{solution-24}}

\begin{Shaded}
\begin{Highlighting}[]
\NormalTok{diamonds }\OperatorTok\StringTok{ }
\StringTok{  }\KeywordTok{ggplot}\NormalTok{(}\DataTypeTok{mapping =} \KeywordTok{aes}\NormalTok{(}\DataTypeTok{x =}\NormalTok{ color, }\DataTypeTok{y =}\NormalTok{ price)) }\OperatorTok{+}
\StringTok{  }\KeywordTok{geom_boxplot}\NormalTok{()}
\end{Highlighting}
\end{Shaded}

\includegraphics{HW_template_01_files/figure-latex/solution_05b-1.pdf}

Discuss your conclusions here.

\hypertarget{c-4}{%
\subsubsection{5c)}\label{c-4}}

Next, we will include the influence of the \texttt{cut} variable
breaking up the graphic into separate subplots based on the levels of
\texttt{cut}.

\hypertarget{problem-26}{%
\paragraph{PROBLEM}\label{problem-26}}

\textbf{Add the \texttt{facet\_wrap()} call to the code from Problem
5b), and set the facetting variable to be \texttt{cut}.}

\hypertarget{solution-25}{%
\paragraph{SOLUTION}\label{solution-25}}

\begin{Shaded}
\begin{Highlighting}[]
\NormalTok{diamonds }\OperatorTok\StringTok{ }
\StringTok{  }\KeywordTok{ggplot}\NormalTok{(}\DataTypeTok{mapping =} \KeywordTok{aes}\NormalTok{(}\DataTypeTok{x =}\NormalTok{ color, }\DataTypeTok{y =}\NormalTok{ price)) }\OperatorTok{+}
\StringTok{  }\KeywordTok{geom_boxplot}\NormalTok{() }\OperatorTok{+}
\StringTok{  }\KeywordTok{facet_wrap}\NormalTok{(}\OperatorTok{~}\StringTok{ }\NormalTok{cut)}
\end{Highlighting}
\end{Shaded}

\includegraphics{HW_template_01_files/figure-latex/solution_05c-1.pdf}

\hypertarget{d-4}{%
\subsubsection{5d)}\label{d-4}}

In addition to the \texttt{facet\_wrap()} function, we can create
subplots with the \texttt{facet\_grid()} function. As the name suggests,
\texttt{facet\_grid()} creates a 2D grid layout where each subplot
corresponds to a combination of two facetting variables. As with
\texttt{facet\_wrap()}, the syntax uses the formula interface:
\texttt{facet\_grid(\textless{}vertical\ variable\textgreater{}\ \textasciitilde{}\ \textless{}horizontal\ variable\textgreater{})}.
The variable provided to the left of the \texttt{\textasciitilde{}}
varies top-to-bottom (vertically), while the variable to the right of
the \texttt{\textasciitilde{}} changes left-to-right (horizontally).

\hypertarget{problem-27}{%
\paragraph{PROBLEM}\label{problem-27}}

\textbf{To see how this works, use \texttt{facet\_grid()} instead of
\texttt{facet\_wrap()} and set the facetting variables to be
\texttt{clarity} and \texttt{cut} for the vertical and horizontal
directions, respectively.}

\hypertarget{solution-26}{%
\paragraph{SOLUTION}\label{solution-26}}

\begin{Shaded}
\begin{Highlighting}[]
\NormalTok{diamonds }\OperatorTok\StringTok{ }
\StringTok{  }\KeywordTok{ggplot}\NormalTok{(}\DataTypeTok{mapping =} \KeywordTok{aes}\NormalTok{(}\DataTypeTok{x =}\NormalTok{ color, }\DataTypeTok{y =}\NormalTok{ price)) }\OperatorTok{+}
\StringTok{  }\KeywordTok{geom_boxplot}\NormalTok{() }\OperatorTok{+}
\StringTok{  }\KeywordTok{facet_grid}\NormalTok{(clarity }\OperatorTok{~}\StringTok{ }\NormalTok{cut)}
\end{Highlighting}
\end{Shaded}

\includegraphics{HW_template_01_files/figure-latex/solution_05d-1.pdf}

\hypertarget{e-4}{%
\subsubsection{5e)}\label{e-4}}

The resulting figure in Problem 5d) includes 3 out of the 4 C's. The
remaining variable, \texttt{carat}, is not categorical. To include
\texttt{carat} in our figure, let's discretize it and compare two
boxplots side-by-side at each \texttt{color} level within each
\texttt{clarity} and \texttt{cut} subplot combination. For now, we will
keep things simple and break up \texttt{carat} based on if an
observation has a value greater than the median \texttt{carat} value.

\hypertarget{problem-28}{%
\paragraph{PROBLEM}\label{problem-28}}

\textbf{Within the \texttt{geom\_boxplot()} function, set the
\texttt{fill} aesthetic to be a conditional test:
\texttt{carat\ \textgreater{}\ median(carat)}. As shown in the
supplemental reading material, use the \texttt{theme()} function to move
the legend position to the top of the graphic.}

\emph{Note}: It might be difficult to see everything within the graphic
window dispalyed in the result after the code chunk, when working within
the .Rmd file in the RStudio IDE. You can zoom in by clicking on the
``Show in New Window'' icon which is displayed as the small ``arrow over
paper'' icon to the right hand size of the output portion.
Alternatively, the figure dimensions can be modified by the code chunk
parameters \texttt{fig.width} and \texttt{fig.height}. \textbf{For this
assignment, it is ok to use the default figure dimensions.}

\hypertarget{solution-27}{%
\paragraph{SOLUTION}\label{solution-27}}

\begin{Shaded}
\begin{Highlighting}[]
\NormalTok{diamonds }\OperatorTok
\StringTok{  }\KeywordTok{ggplot}\NormalTok{(}\DataTypeTok{mapping =} \KeywordTok{aes}\NormalTok{(}\DataTypeTok{x =}\NormalTok{ color, }\DataTypeTok{y =}\NormalTok{ price)) }\OperatorTok{+}
\StringTok{  }\KeywordTok{geom_boxplot}\NormalTok{(}\DataTypeTok{mapping =} \KeywordTok{aes}\NormalTok{(}\DataTypeTok{fill =}\NormalTok{ carat }\OperatorTok{>}\StringTok{ }\KeywordTok{median}\NormalTok{(carat))) }\OperatorTok{+}
\StringTok{  }\KeywordTok{facet_grid}\NormalTok{(clarity }\OperatorTok{~}\StringTok{ }\NormalTok{cut) }\OperatorTok{+}
\StringTok{  }\KeywordTok{theme}\NormalTok{(}\DataTypeTok{legend.position =} \StringTok{'top'}\NormalTok{)}
\end{Highlighting}
\end{Shaded}

\includegraphics{HW_template_01_files/figure-latex/solution_05e-1.pdf}

\hypertarget{f-2}{%
\subsubsection{5f)}\label{f-2}}

Due to the large number of subplots, the individual facets are quite
small with the default figure size. Let's focus on the case with
\texttt{cut\ ==\ "Ideal"} and \texttt{clarity\ ==\ "IF"} by calling
\texttt{filter()} before piping the dataset into the \texttt{ggplot()}.

\hypertarget{problem-29}{%
\paragraph{PROBLEM}\label{problem-29}}

\textbf{Pipe \texttt{diamonds} into \texttt{filter()} and perform the
necessary operation. Pipe the resulting dataset into the same
\texttt{ggplot2} function calls used in Problem 5e), except for one
important change. The \texttt{filter()} call will reduce the dataset,
and thus our conditional test will be comparing \texttt{carat} to the
median value associated with the smaller dataset. To force the
conditional test to still be applied to the median based on the complete
dataset use \texttt{median(diamonds\$carat)} within the conditional test
instead of \texttt{median(carat)}.}

\hypertarget{solution-28}{%
\paragraph{SOLUTION}\label{solution-28}}

\begin{Shaded}
\begin{Highlighting}[]
\NormalTok{diamonds }\OperatorTok
\StringTok{  }\KeywordTok{filter}\NormalTok{(cut }\OperatorTok{==}\StringTok{ 'Ideal'}\NormalTok{, clarity }\OperatorTok{==}\StringTok{ 'IF'}\NormalTok{) }\OperatorTok
\StringTok{  }\KeywordTok{ggplot}\NormalTok{(}\DataTypeTok{mapping =} \KeywordTok{aes}\NormalTok{(}\DataTypeTok{x =}\NormalTok{ color, }\DataTypeTok{y =}\NormalTok{ price)) }\OperatorTok{+}
\StringTok{  }\KeywordTok{geom_boxplot}\NormalTok{(}\DataTypeTok{mapping =} \KeywordTok{aes}\NormalTok{(}\DataTypeTok{fill =}\NormalTok{ carat }\OperatorTok{>}\StringTok{ }\KeywordTok{median}\NormalTok{(diamonds}\OperatorTok{$}\NormalTok{carat))) }\OperatorTok{+}
\StringTok{  }\KeywordTok{facet_grid}\NormalTok{(clarity }\OperatorTok{~}\StringTok{ }\NormalTok{cut) }\OperatorTok{+}
\StringTok{  }\KeywordTok{theme}\NormalTok{(}\DataTypeTok{legend.position =} \StringTok{'top'}\NormalTok{)}
\end{Highlighting}
\end{Shaded}

\includegraphics{HW_template_01_files/figure-latex/solution_05f-1.pdf}

\hypertarget{g-1}{%
\subsubsection{5g)}\label{g-1}}

\hypertarget{problem-30}{%
\paragraph{PROBLEM}\label{problem-30}}

\textbf{Discuss the differences between the trends shown in the
resulting figure in Problem 5f) with the trends shown in the figure in
Problem 5b).}

\hypertarget{solution-29}{%
\paragraph{SOLUTION}\label{solution-29}}

The trend shown in 5b mainly shows us how color affects the price of a
diamond. Yet according to the box plot in 5b, color is not a dominant
factor of price. All 7 color have similar price ranges. 5f on the other
hand, shows us that the real determinant factor of price is carat,
specifically as in IF cut. As it clearly shows that diamonds that have
less carats are significant cheaper than those with more carats.

\hypertarget{problem-6-1}{%
\subsection{Problem 6}\label{problem-6-1}}

We can create LaTeX style math expressions within R Markdown. To get
more comfortable writing these type of expressions, we will work
examples similar to those discussed in the calculus review in the first
week of lecture.

\hypertarget{quick-introduction-to-latex}{%
\paragraph{Quick introduction to
LaTeX}\label{quick-introduction-to-latex}}

In R Markdown, math notation and expressions can be displayed inline by
typing expressions between dollar signs. For example, typing
\texttt{\$x\$} displays \(x\) in the rendered document. If we instead
want to display a bold face variable (to represent a vector) we will
need to include the LaTeX style function
\texttt{\textbackslash{}mathbf()} between the dollar signs. Thus, typing
\texttt{\$\textbackslash{}mathbf\{x\}\$} renders \(\mathbf{x}\) in the
document.

LaTeX style expressions provide a lot of flexibility and enable us to
write out expressions just like we would on paper. To show a subscript
``attached'' or ``associated'' with variable such as \(x_1\), you must
use the underscore character \texttt{\_} next to that variable. Thus,
type \texttt{\$x\_1\$} to display \(x_1\). If you have a relatively
complicated expression you want to show within the subscript, you can
wrap curly braces \texttt{\{\}} around the subscript expression. For
example to display \(x_{i,j,k,l}\) simply type
\texttt{\$x\_\{i,j,k,l\}\$}.

Superscripts, such as \(x^2\), are displayed with the \texttt{\^{}}
operator. Thus, to show the square of the variable \(x\), type
\texttt{\$x\^{}2\$}. As with subscripts, you can wrap complicated
expressions within curly braces. To show \(x^{t+2}\), type
\texttt{\$x\^{}\{t+2\}\$}. Subscripts and superscripts can also be
combined together, such as with \(x_{n}^{2}\). When combining subscripts
and superscripts, it's recommended to wrap each expression within curly
braces, even if the expression is simple. Therefore, to display
\(x_{n}^{2}\), type \texttt{\$x\_\{n\}\^{}\{2\}\$}.

We can display variables and complete expressions within parentheses,
such as \(\left(x+a\right)\), several different ways. First, you can
simply type \texttt{\$(x\ +\ a)\$} which produces \((x + a)\). Another
option though is to use the following commands
\texttt{\$\textbackslash{}left(\ x\ +\ a\ \textbackslash{}right)\$}
which also produces \(\left( x + a \right)\). At first glance these seem
to be the same, but the
\texttt{\textbackslash{}left(\ \textbackslash{}right)} commands allow
the parantheses to be dynamically sized around the expression they
contain. For more information on different types of brackets and
parantheses, please see the following reference from
\href{https://www.overleaf.com/learn/latex/Brackets_and_Parentheses}{Overleaf}.

We can also display equations within ``equation blocks'' using two
dollar signs to surround the expression of interest. The equation within
the block is rendered beneath text in the rendered report. We can thus
draw special focus to mathematical equations and expressions displayed
this way.

As an example, consider a function of two variables, \(x_1\) and
\(x_2\), shown in the equation block below:

\[ 
f\left(x_1, x_2 \right) = a x_{1}^{b} + c x_{2}^{d}
\]

To create the above equation, type the following:

\texttt{\$\$\ f\textbackslash{}left(x\_1,\ x\_2\ \textbackslash{}right)\ =\ a\ x\_\{1\}\^{}\{b\}\ +\ c\ x\_\{2\}\^{}\{d\}\ \$\$}

Reading the un-rendered LaTeX expression may seem daunting at first, and
getting used to working with LaTeX math coding can be challenging.
However, please note that the above expression combines each of the
previously described elements together. Each individual element, such as
a subscript or superscript is relatively straight forward. Thus, when
starting out with LaTeX, break a complicated expression into separate
components or elements. Get those elements correct, and then
``assemble'' them into the final expression of interest.

Let's now get additional practice by working through the partial
derivatives of the above function.

\hypertarget{a-5}{%
\subsubsection{6a)}\label{a-5}}

To write out the derivatives, we need to understand how to type
fractions in LaTeX notation. A fraction consists of three parts. The
first is the phrase \texttt{\textbackslash{}frac}. The remaining two
parts of a fraction are two sets of curly braces. The complete syntax is
\texttt{\textbackslash{}frac\{\}\{\}}. The numerator is typed within the
first set of curly braces and the denominator is typed within the second
set of curly braces. Thus, the formulation for a fraction is:

\texttt{\textbackslash{}frac\{\textless{}numerator\ expression\textgreater{}\}\{\textless{}denominator\ expression\textgreater{}\}}.

As an example, one half written as a fraction is
\texttt{\$\textbackslash{}frac\{1\}\{2\}\$}. The rendered LaTeX
expression looks like \(\frac{1}{2}\).

\hypertarget{problem-31}{%
\paragraph{PROBLEM}\label{problem-31}}

\textbf{Write out the partial derivative of \(f\) with respect to
\(x_1\). To receive full credit you must use the
\texttt{\textbackslash{}frac\{\}\{\}} to write out the partial
derivative of \(f\) with respect to \(x_1\) on the left hand side of
\(=\). The right hand side of \(=\) must be your expression for the
partial first derivative.}

\textbf{NOTE:} The ``partial'' operator, \(\partial\) is created by
typing \texttt{\textbackslash{}partial}.

\hypertarget{solution-30}{%
\paragraph{SOLUTION}\label{solution-30}}

\[ 
\frac{\partial{f}}{\partial{$x_1$}} = ab$x_1$^{b - 1}
\]

\hypertarget{b-5}{%
\subsubsection{6b)}\label{b-5}}

Now write out the expression for the partial derivative of \(f\) with
respect to \(x_2\).

\hypertarget{problem-32}{%
\paragraph{PROBLEM}\label{problem-32}}

\textbf{Write out the partial derivative of \(f\) with respect to
\(x_2\). To receive full credit you must use the
\texttt{\textbackslash{}frac\{\}\{\}} to write out the partial
derivative of \(f\) with respect to \(x_2\) on the left hand side of
\(=\). The right hand side of \(=\) must be your expression for the
partial first derivative.}

\textbf{NOTE:} The ``partial'' operator, \(\partial\) is created by
typing \texttt{\textbackslash{}partial}.

\hypertarget{solution-31}{%
\paragraph{SOLUTION}\label{solution-31}}

\[ 
HERE
\]

\hypertarget{c-5}{%
\subsubsection{6c)}\label{c-5}}

As discussed in lecture, we also need to work with second derivatives.

\hypertarget{problem-33}{%
\paragraph{PROBLEM}\label{problem-33}}

\textbf{Write out the partial second derivative of \(f\) with respect to
\(x_1\). Remember that the second derivative is the derivative of the
first derivative.}

\hypertarget{solution-32}{%
\paragraph{SOLUTION}\label{solution-32}}

\[ 
HERE
\]

\hypertarget{d-5}{%
\subsubsection{6d)}\label{d-5}}

The second derivative in 6c) represents the rate-of-change with respect
to \(x_1\) of the rate-of-change with respect to \(x_1\). We can also
consider the cross-derivative, which examines the rate-of-change with
respect to \(x_2\) of the rate-of-change with respect to \(x_1\).

\hypertarget{problem-34}{%
\paragraph{PROBLEM}\label{problem-34}}

\textbf{What is the cross derivative equal to, for the example function
of two variables? Write out the expression within an equation block
below.}

\hypertarget{solution-33}{%
\paragraph{SOLUTION}\label{solution-33}}

\[ 
HERE
\]

\hypertarget{e-5}{%
\subsubsection{6e)}\label{e-5}}

In lecture, we also reviewed important linear algebra concepts. We
discussed the inner product, and how we can write it out two ways. One
approach is with vector-vector math, while the other involves a
summation. For the next several problems you will work with a vector
\(\mathbf{x}\) which is a \(\left(N \times 1\right)\) column vector.

\hypertarget{problem-35}{%
\paragraph{PROBLEM}\label{problem-35}}

\textbf{Write out the vector-vector operation for the inner product of
the vector \(\mathbf{x}\) with itself. Place your expression within an
equation block below. You do not need to write out the elements of the
vectors.}

\hypertarget{solution-34}{%
\paragraph{SOLUTION}\label{solution-34}}

\[ 
HERE
\]

\hypertarget{f-3}{%
\subsubsection{6f)}\label{f-3}}

The summation expression displayed in the instructions for Problem 6f)
consists of three components. The first is the Sigma character, which is
created by the typing \texttt{\textbackslash{}sum}. If we want the sigma
character to appear in line, we wrap dollar signs around it,
\texttt{\$\textbackslash{}sum\$}, and if we want it to appear within an
equation block we wrap two dollar signs,
\texttt{\$\$\textbackslash{}sum\$\$}. The \(n=1\) and \(N\) characters
are displayed below and above the Sigma character with the \texttt{\_}
and \texttt{\^{}} special characters, respectively.

\hypertarget{problem-36}{%
\paragraph{PROBLEM}\label{problem-36}}

\textbf{Write out the summation approach to the inner product within an
equation block below.}

\hypertarget{solution-35}{%
\paragraph{SOLUTION}\label{solution-35}}

\[ 
HERE
\]

\hypertarget{g-2}{%
\subsubsection{6g)}\label{g-2}}

In the 6e), the vector \(\mathbf{x}\) was defined to be a
\(\left(N \times 1 \right)\) column vector. What are the dimensions of
the inner product of \(\mathbf{x}\) with itself? What would the
dimensions of the outer product of \(\mathbf{x}\) be with itself?

\hypertarget{problem-37}{%
\paragraph{PROBLEM}\label{problem-37}}

\textbf{Write the dimensions of the inner product and the dimensions of
the outer product of \(\mathbf{x}\) with itself. Note that to display
the multiplication sign you can type \texttt{\textbackslash{}times}.}

\hypertarget{solution-36}{%
\paragraph{SOLUTION}\label{solution-36}}

Type your answer here.

\hypertarget{h}{%
\subsubsection{6h)}\label{h}}

We will frequently use greek letters when writing expressions and
equations. Thankfully, it is rather simple to type greek letters in
LaTeX. All you have to do is type the escape character
\texttt{\textbackslash{}} in front of the greek word. For example to
display \(\delta\), just type \texttt{\$\textbackslash{}delta\$}. To
show the capital letter \(\Delta\) use a capital \texttt{D} instead of a
lower case \texttt{d}: \texttt{\$\textbackslash{}Delta\$}.

\hypertarget{problem-38}{%
\paragraph{PROBLEM}\label{problem-38}}

\textbf{Practice typing the greek letters listed in the instruction file
in an equation block below. Separate each letter by the LaTeX term
\texttt{\textbackslash{}cdot} which displays the ``dot-multiply''
operator, \(\cdot\).}

\emph{Hint}: the greek words are alpha, beta, gamma, epsilon, mu, sigma,
and psi.

\hypertarget{solution-37}{%
\paragraph{SOLUTION}\label{solution-37}}

\[ 
HERE
\]

\end{document}
